\documentclass[sigplan, review]{acmart}

\usepackage{booktabs} % For formal tables


% Copyright
%\setcopyright{none}
%\setcopyright{acmcopyright}
%\setcopyright{acmlicensed}
\setcopyright{rightsretained}
%\setcopyright{usgov}
%\setcopyright{usgovmixed}
%\setcopyright{cagov}
%\setcopyright{cagovmixed}


% DOI
%\acmDOI{10.475/123_4}

% ISBN
%\acmISBN{123-4567-24-567/08/06}

%Conference
\acmConference[DHPCC'2017]{Distributed \& Heterogeneous Programming in
  C/C++ Workshop 2017}{May 2017}{Toronto, Canada}

\acmMonth{5}
\acmYear{2017}

%\acmPrice{15.00}

%\acmBadgeL[http://ctuning.org/ae/ppopp2016.html]{ae-logo}
%\acmBadgeR[http://ctuning.org/ae/ppopp2016.html]{ae-logo}


\begin{document}

\title{SYCL C++17 and OpenCL interoperability with triSYCL}

\author{Anastasios Doumoulakis}
\email{anastasi@xilinx.com}

\author{Ronan Keryell}
\email{Ronan.Keryell@xilinx.com}

\author{Kenneth O'Brien}
\email{kennetho@xilinx.com}

\affiliation{%
  \institution{Xilinx Research}
  \streetaddress{2020 Bianconi Avenue}
  \city{Dublin}
  \country{Ireland}
  \postcode{D24 T683}
}


\begin{abstract}
Heterogeneous computing 
\end{abstract}

%
% The code below should be generated by the tool at
% http://dl.acm.org/ccs.cfm
% Please copy and paste the code instead of the example below. 
%
\begin{CCSXML}
<ccs2012>
 <concept>
  <concept_id>10010520.10010553.10010562</concept_id>
  <concept_desc>Computer systems organization~Embedded systems</concept_desc>
  <concept_significance>500</concept_significance>
 </concept>
 <concept>
  <concept_id>10010520.10010575.10010755</concept_id>
  <concept_desc>Computer systems organization~Redundancy</concept_desc>
  <concept_significance>300</concept_significance>
 </concept>
 <concept>
  <concept_id>10010520.10010553.10010554</concept_id>
  <concept_desc>Computer systems organization~Robotics</concept_desc>
  <concept_significance>100</concept_significance>
 </concept>
 <concept>
  <concept_id>10003033.10003083.10003095</concept_id>
  <concept_desc>Networks~Network reliability</concept_desc>
  <concept_significance>100</concept_significance>
 </concept>
</ccs2012>  
\end{CCSXML}

\ccsdesc[500]{Computer systems organization~Embedded systems}
\ccsdesc[300]{Computer systems organization~Redundancy}
\ccsdesc{Computer systems organization~Robotics}
\ccsdesc[100]{Networks~Network reliability}

% We no longer use \terms command
%\terms{Theory}

\keywords{C++17, SYCL, OpenCL}


\maketitle

\section{Introduction}
\label{sec:introduction}

Computing architectures nowadays are huge multi-processor
system-on-chips with different kind of processors, GPU, configurable
specific accelerators (video CODEC...), reconfigurable programmable
logic (FPGA), various hierarchies of memory and memory interfaces,
configurable IO and network support, security support, power control,
etc. High-performance applications may use a hierarchy of such system
up to fill up a full-scale data-center.

So the programmer is facing nowadays a fractal architecture, demanding
also more and more control for power efficiency. This tends to require
a dense fractal set of skills and tools.

SYCL \cite{C++:P00236R0:SYCL} is a new open standard from Khronos
Group aiming at solving some of the programming issues related to
heterogeneous computing.  This pure C++17 domain-specific embedded
language allows the programmer to write single-source C++17 host code
with accelerated code expressed as functors. The data accesses are
described with accessor objects that implicitly define a task graph
that can be asynchronously scheduled on a distributed-memory system
including several CPU and accelerators.

This programming model is quite generic but provides also an
interoperability mode with the OpenCL realm, another standard from
Khronos Group aimed at heterogeneous computing with a C host API and
separate language for the kernels (C, C++, SPIR and SPIR-V).  This
allows a SYCL C++ application to recycle existing OpenCL kernels into
a higher level C++ programming model, relieving the programmer from
explicitly defining the memory transfers.

In this article we present in Section~\ref{sec:sycl} the SYCL
standard, then in Section~\ref{sec:sycl-opencl-inter} to finish in
Section~\ref{sec:exper-with-opencl} with some experiments with the
triSYCL open source implementation of the SYCL standard.

\section{SYCL}
\label{sec:sycl}


\section{SYCL and OpenCL interoperability mode}
\label{sec:sycl-opencl-inter}


\section{Experimenting with OpenCL interoperability mode of SYCL}
\label{sec:exper-with-opencl}


\section{Conclusion}
\label{sec:conclusion}


\bibliographystyle{ACM-Reference-Format}
\bibliography{biblio}

\end{document}

%%% Local Variables:
%%% mode: latex
%%% TeX-master: t
%%% TeX-auto-untabify: t
%%% TeX-PDF-mode: t
%%% ispell-local-dictionary: "american"
%%% End:
